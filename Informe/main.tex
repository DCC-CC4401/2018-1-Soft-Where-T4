% Template:     Informe/Reporte LaTeX
% Documento:    Archivo principal
% Versión:      4.7.3 (05/02/2018)
% Codificación: UTF-8
%
% Autor: Pablo Pizarro R.
%        Facultad de Ciencias Físicas y Matemáticas
%        Universidad de Chile
%        pablo.pizarro@ing.uchile.cl, ppizarror.com
%
% Manual template: [http://latex.ppizarror.com/Template-Informe/]
% Licencia MIT:    [https://opensource.org/licenses/MIT/]

% CREACIÓN DEL DOCUMENTO
\documentclass[letterpaper,11pt]{article} % Articulo tamaño carta, 11pt
\usepackage[utf8]{inputenc} % Codificación UTF-8

% INFORMACIÓN DEL DOCUMENTO
\def\titulodelinforme {Tarea 4}
\def\temaatratar {Entrega Final}

\def\autordeldocumento {Grupo: Soft, where?}
\def\nombredelcurso {Ingeniería de Software I}
\def\codigodelcurso {CC4401}

\def\nombreuniversidad {Universidad de Chile}
\def\nombrefacultad {Facultad de Ciencias Físicas y Matemáticas}
\def\departamentouniversidad {Departamento de Ciencias de la Computación}
\def\imagendepartamento {departamentos/dcc}
\def\imagendepartamentoescala {0.2}
\def\localizacionuniversidad {Santiago, Chile}

% INTEGRANTES, PROFESORES Y FECHAS
\def\tablaintegrantes {
\begin{tabular}{ll}
	Integrantes:
		& \begin{tabular}[t]{@{}l@{}}
			Pedro Belmonte \\
			Ricardo Cordova \\
			Israel Peña \\
			Joaquín Pérez \\
			Heinrich Porro \\
			Matías Villegas
		\end{tabular} \\
	Profesora:
		& \begin{tabular}[t]{@{}l@{}}
			Jocelyn Simmonds
		\end{tabular} \\
	Auxiliar:
		& \begin{tabular}[t]{@{}l@{}}
			Constanza Escobar
		\end{tabular} \\
	Ayudantes:
		& \begin{tabular}[t]{@{}l@{}}
			Sergio Leiva \\
			Pablo Miranda
		\end{tabular} \\
	\multicolumn{2}{l}{\localizacionuniversidad}
\end{tabular}
}

% CONFIGURACIONES
\input{lib/config}

% IMPORTACIÓN DE LIBRERÍAS
\input{lib/imports}

% IMPORTACIÓN DE FUNCIONES
\input{lib/function/core}
\input{lib/function/elements}
\input{lib/function/equation}
\input{lib/function/image}
\input{lib/function/title}

% IMPORTACIÓN DE ENTORNOS
\input{lib/environments}

% IMPORTACIÓN DE ESTILOS
\input{lib/styles}

% CONFIGURACIÓN INICIAL DEL DOCUMENTO
\input{lib/initconf}

\usepackage{hyperref} %%<-
\newcommand{\tabitem}{~~\llap{\textbullet}~~}

% INICIO DE LAS PÁGINAS
\begin{document}

% PORTADA
\input{lib/portrait}

% CONFIGURACIÓN DE PÁGINA Y ENCABEZADOS
\input{lib/pageconf}

% RESUMEN O ABSTRACT

% TABLA DE CONTENIDOS - ÍNDICE1

% CONFIGURACIONES FINALES
\input{lib/finalconf}

% ======================= INICIO DEL DOCUMENTO =======================

% Leave this code alone! just compile...
\section{Introducción}
En esta última iteración del trabajo pedido, a partir del código realizado por el grupo seleccionado de la tarea anterior, se debe concebir un producto funcional, además de la elaboración de un plan de pruebas para verificar las diferentes funcionalidades que están siendo pedidas dentro de los requisitos del sistema. \\

En este informe, se presentan las metodologías de trabajo usadas por Soft-Where para realizar el trabajo descrito anteriormene, luego se presentra el Plan de Pruebas realizado por el equipo para corroborar que los requisitos del sistema se están cumpliendo y detectar las posibles fallas durante el desarrollo, posteriormente se describirán los cambios realizados por el equipo al sistema realizado por el equipo ganador del trabajo anterior, concluyendo este informe con una reflexión sobre el trabajo realizado.



\iffalse
% Texto para motivarse.
En la tercera iteración del trabajo realizado, se deben desarrollar las interfaces diseñadas para el sistema, por el grupo seleccionado en la tarea anterior. \\
Las interfaces solicitadas corresponden al landing page para personas naturales, el landing page para administradores, el perfil del usuario (visto por el dueño del perfil) y la ficha de un artículo. Para cada uno de ellos, se utiliza el diseño realizado por Soft-of-War (según lo votado por el curso). \\
En este informe, se presenta las metodologías utilizadas por el equipo de Soft-Where para realizar estas implementaciones, el modelo de datos utilizado y su relación con las interfaces gráficas implementadas; finalmente, se presenta como las interfaces implementadas satisfacen los requisitos funcionales establecidos para esta entrega. 
Se concluye este informe con reflexiones sobre el trabajo y la metodología utilizada.
\fi

\newpage
\section{Metodología de Equipo}
La metodología utilizada por el equipo fue [...]

% Dejo la metodología anterior para la inspiración posterior.
\iffalse

El equipo de trabajo utilizó una metodología de trabajo tipo Scrum: esto es, con trabajo diario de avance en "features", divididos para los distintos miembros del equipo, necesarios para la implementación de los requisitos, con revisión constante de los logros conseguidos. \\
Sin embargo, hay ciertas limitantes que no permiten realizar un trabajo Scrum a cabalidad: no es posible hacer reuniones diarias dada la distancia entre los miembros, así como los horarios de la universidad, además de que no se dispone de los espacios necesarios como para poder llevar una tabla de avances como lo requiere esta metodología de trabajo. Herramientas como Git y Git Flow ayudan a suplir algunas de estas falencias, ya que permiten ir revisando los avances que realizan los otros miembros del equipo y los "features" sobre los que trabajan.\\
El desarrollo de las implementaciones se realiza utilizando el Framework de Django, que permite levantar el funcionamiento de un servicio web rápidamente con backend basado en Python y frontend basado en html y sintaxis de DTL que separa el funcionamiento del diseño, permitiendo que cada parte pueda trabajar de forma independiente sin tener que repetir información de uno en otro. Otros elementos que se añaden para mejorar el trabajo desarrollado son Bootstrap, para tener un diseño de web más responsivo, y Pillow, para facilitar el manejo de imágenes en la base de datos. \\
Dada esta facilidad de separar las partes, y las restricciones a las que nos enfrentamos, se decide que los miembros del equipo deberían trabajar en las áreas que consideren son las más fuertes para ellos mismos. De esta forma, se aprovechan las habilidades y los intereses de cada uno de los miembros del equipo. \\
Por ejemplo, Ricardo trabajó principalmente en Backend, en lo referido al modelo de datos y el sistema de login y creación de usuarios; mientras que Pedro trabajó principalmente en el Frontend, pues conocía el funcionamiento de Bootstrap y tiene buenas nociones de diseño. De está forma, cada uno de los miembros del equipo pudo aportar lo mejor de cada uno. \\

% La tabla más dificil de rellenar que la coevaluación...

\begin{table}[H]
  \centering
	\begin{tabular}{|c|c|}
	\hline 
	Miembro Equipo & Participación \\ 
	\hline 
	Pedro Belmonte & 
	Principalmente Front-End, Grilla de Espacios, Landing Page User
	\\ 
	\hline 
	Ricardo Cordova & 
	Principalmente Back-End, Sistema de Registros/Login, Permisos de Administrador
	\\ 
 	\hline 
	Israel Peña & 
	Pedidos de Artículos, Ficha de Artículo, Admin Landing Page
	\\ 
	\hline 
	Joaquín Pérez &
	Perfil de Usuario, Diferenciación del Landing Page, Realizar Merging de los commit 
	\\ 
	\hline 
	\end{tabular} 
\end{table}

Gracias a esta metodología de trabajo, se pudieron implementar los requerimientos indicados por lo solicitado en las instrucciones de la tarea n° 3: los requisitos 1-2, 4-5, 7-8, 12, 14-18, 23, 25-26, 28, 37-40, 51 y 54-56. En la sección 4 de este informe se especifíca como se satisfacen estos requisitos mediante las interfaces implementadas. En la sección 3 se observara el modelo de datos que subyace a las interfaces implementadas, y como estas se relacionan con los modelos para poder implementar sin tener que re-escribir código. 
\fi

\newpage
\section{Plan de Testing}
[Texto explicativo de como se diseño el testing]
% 
\begin{itemize}
	\item \textbf{ID}: 01 \\
		\textbf{Descripción}: Prueba el Registro de usuarios nuevos. \\
		\textbf{Requisitos a probar}: 1.\\
		\textbf{Inicialización}: El usuario que se pretende registrar no debe estar registrado previamente.\\
		\textbf{Pasos de la prueba}: En la página de login se debe hacer click en el enlace que dice "¿No tienes cuenta? Crea una.", luego rellenar el el formulario con un correo electrónico, un nombre, un apellido, un RUT y una contraseña, finalmente hacer click en el botón "Crear cuenta".\\
		\textbf{Resultados esperados}: El sistema no debería registrar la cuenta hasta que no se llenen todos los campos, adicionalmente se agrega el usuario nuevo a la base de datos y se redirige a la landing page para personas naturales.\\
\end{itemize}

\newpage
\section{Cambios al Sistema}



\iffalse
El modelo de datos es el siguiente: \\
\begin{figure}[H]
\includegraphics[width=0.5\textheight]{images/modelo_de_datos.png}
\caption{Diagrama del modelo de datos implementado} \label{LPUser}
\end{figure}
En este modelo se pueden distinguir tres niveles: un primer nivel generado automaticamente por Django en cuanto se utilizan sus modelos de Groups, Permission y User; un segundo nivel creado por el equipo donde se encuentran los elementos directamente relacionados con los requisitos que son los Artículos, Usuarios y Espacios; el último nivel es creado también por el equipo, pero son tablas relacionales entre los usuarios y artículos y espacios que representan los pedidos realizados por los usuarios y los resultados históricos de estas solicitudes.
Los modelos del primer nivel permiten generar un sistema de creación de usuario y logín con facilidad. Estos modelos disponen de nombre, apellido, correo, clave, permisos y grupos, todo ello por defecto dentro de Django. Así, podemos diferenciar una Persona Natural de un Administrador utilizando grupos, dando permisos por separado a cada grupo según lo indicado en los requisitos, así como poder guardar la información requerida por los usuarios. \\
Sin embargo, se hace necesario complementar esta información con datos adicionales que no corresponden al modelo de Django. Para ello, en el segundo nivel del modelo se extiende el usuario por defecto de Django con un usuario personalizado con nueva información adicional, como el rut y la foto de perfil; esta información servira para poblar de contenido el perfil del usuario. Además, consideramos tablas adicionales para los artículos y los espacios que dispondra el sistema para prestar a sus usuarios. La información que ahí se almacena servira para poder poblar las fichas de los artículos y espacios. \\
El último nivel, que posee las tablas relacionales entre usuario y artículo, y usuario y espacio; estas tablas almacenan la información de los pedidos hechos por un usuario, la fecha de pedido y devolución. Esta información se puede usar por el perfil del usuario como un registro histórico de sus pedidos, además de poder indicar al sistema que elementos se encuentran disponibles para que otros usuarios puedan pedir o no un artículo (según las fechas de los pedidos), así como informar a los administradores sobre la vigencia o caducidad de una solicitud (entre otros posibles). 
\fi

\newpage
\section{Conclusión}

%Igual aquí...
\iffalse
Luego de haber realizado la implementación de las interfaces del sistema estudiado durante el curso, se pueden llegar a una serie de conclusiones, en base al trabajo realizado, las herramientas utilizadas y las metodologías aplicadas.\\
En cuanto al trabajo realizado, podemos ver como la organización por requisitos ayuda a organizar los objetivos y la funcionalidad deseada para una aplicación, eliminando las distracciones y las ideas que pueden surgir a medida que se implementan otras funcionalidades. De esta forma, el trabajo se hace más organizado, más directo y más sencillo para todos los que participan. Por otro lado, muchas veces los requisitos implican otras funcionalidades, lo cual complica los límites y las responsabilidades de los trabajadores: por ejemplo, si una persona se encuentra encargada de desarrollar modelos, y otra se encuentra encargada de desarrollar un sistema de login, ambas partes se relacionan con los permisos de usuarios y la posibilidad de hacer préstamos. ¿Quién es el responsable de que el usuario pueda, efectivamente, hacer el pedido? \\
Las herramientas utilizadas, en particular Git, ayudan a superar estos problemas y ha hacer que el trabajo sea más armonioso. Si una parte no está resuelta o si falta funcionalidad, los responsables de las partes individuales pueden compartir sus inquietudes y resolver aquello que impide que su código funcione. De la misma forma, Django ayuda a separar las responsabilidades, asegurando que gran parte del sistema pueda funcionar de forma autonoma a las otras (es decir, desarrollar un frontend independiente del desarrollo del backend, pero que asume que el otro estará funcionando para poder tener en "andando" el sistema). \\
Finalmente, la metodología de trabajo tipo Scrum que se utilizó ayuda a acelerar los resultados deseados del trabajo. Poder separar el trabajo no solo en fronend y backend, si no que en "features", aseguraba el avance constante del quehacer de cada uno de los miembros del trabajo.
\fi

% FIN DEL DOCUMENTO
\end{document}
